\documentclass{article}
\usepackage[utf8]{inputenc}
\usepackage{graphicx}
\usepackage{hyperref}
\usepackage{geometry}
\geometry{a4paper, margin=1in}

\title{Curiosity Article 2}
\author{Your Name}
\date{\today}

\begin{document}

\maketitle

\section{Introduction}
In this article, we explore an intriguing curiosity that has captured the attention of many. The aim is to delve into the details and provide insights that are both informative and engaging.

\section{Main Content}
Here, we discuss the main aspects of the curiosity. This could include historical context, scientific explanations, or anecdotal evidence that supports the topic. Use figures and tables where necessary to enhance understanding.

\begin{figure}[h]
    \centering
    \includegraphics[width=0.7\textwidth]{path/to/image.jpg}
    \caption{An illustrative image related to the topic.}
    \label{fig:curiosity}
\end{figure}

\section{Conclusion}
In conclusion, this curiosity not only piques interest but also encourages further exploration and understanding. It serves as a reminder of the wonders that exist in our world.

\section{References}
\begin{thebibliography}{9}
\bibitem{reference1}
Author, A. (Year). \textit{Title of the reference}. Publisher.

\bibitem{reference2}
Author, B. (Year). \textit{Title of another reference}. Publisher.
\end{thebibliography}

\end{document}