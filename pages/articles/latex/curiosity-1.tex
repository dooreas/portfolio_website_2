\documentclass{article}
\usepackage[utf8]{inputenc}
\usepackage{geometry}
\usepackage{graphicx}
\usepackage{hyperref}
\usepackage{titlesec}
\usepackage{color}

% Page layout
\geometry{a4paper, margin=1in}

% Title formatting
\titleformat{\section}{\large\bfseries\color{blue}}{}{0em}{} 
\titleformat{\subsection}{\bfseries\color{darkgray}}{}{0em}{}

% Color definitions
\definecolor{darkgray}{rgb}{0.5, 0.5, 0.5}

\title{Curiosity Article Title}
\author{Your Name}
\date{\today}

\begin{document}

\maketitle

\begin{abstract}
This article explores an intriguing curiosity that captures the imagination and sparks interest in the world around us. 
\end{abstract}

\section{Introduction}
In this section, we introduce the topic of curiosity and its significance in our daily lives. Curiosity drives innovation, learning, and exploration.

\section{Main Content}
Here, we delve into the details of the curiosity being discussed. This could include historical context, scientific explanations, or personal anecdotes that relate to the topic.

\subsection{Subtopic 1}
Discuss a specific aspect of the curiosity, providing insights and examples to illustrate your points.

\subsection{Subtopic 2}
Continue exploring the topic, perhaps comparing it to other curiosities or discussing its implications in a broader context.

\section{Conclusion}
Summarize the key points made in the article and reflect on the importance of curiosity in fostering knowledge and understanding.

\section{References}
Include any references or sources that were cited in the article, formatted appropriately.

\end{document}